%===============================================================================
% Finding Significant Subsequences
% Kevin Durant
% March 2019
%===============================================================================

\documentclass[11pt,a4paper]{article}

\usepackage{mathtools}
\usepackage{iftex}
\ifPDFTeX
  \usepackage[T1]{fontenc}
  \usepackage[utf8]{inputenc}
\else
  \usepackage{unicode-math}
\fi
\usepackage{hyperref}

% Custom maths commands %=======================================================

\newcommand\ds{\displaystyle}                 % large maths.
\newcommand\ts{\textstyle}                    % small maths.
\newcommand\mb[1]{\mathbb{#1}}                % mathbb shorthand.
\newcommand\mc[1]{\mathcal{#1}}               % mathcal shorthand.
\newcommand\ub[1]{\symbf{#1}}                 % unicode-math symbf shorthand.
\newcommand\ff[1]{^{\underline{#1}}}          % falling factorial.
\newcommand\rf[1]{^{\overline{#1}}}           % rising factorial.
\newcommand\ul[1]{\underline{#1}}             % underline.
\newcommand\ol[1]{\overline{#1}}              % overline.
\DeclareMathOperator\Pb{P}                    % probability.
\DeclareMathOperator\Ex{E}                    % expected value.
\DeclareMathOperator\Va{V}                    % variance.
\DeclarePairedDelimiter\lr{\lparen}{\rparen}  % sized parentheses.
\DeclarePairedDelimiter\lrb{\lbrack}{\rbrack} % sized brackets.
\DeclarePairedDelimiter\abs{\lvert}{\rvert}   % absolute value symbol.
\DeclarePairedDelimiter\cl{\lceil}{\rceil}    % ceiling symbol.
\DeclarePairedDelimiter\fl{\lfloor}{\rfloor}  % floor symbol.

% Title %=======================================================================

\title{Finding Significant Subsequences}
\author{Kevin Durant}
\date{March 2019}

% Document %====================================================================

\begin{document}

\maketitle

\section{Introduction} %========================================================

A year or two ago, while working on a problem involving temporal communication
networks (in which the underlying edge events are all timestamped, and thus
ordered), a peripheral question arose that seemed worth trying to answer:
\textit{are there any directed paths within the network that are particularly
predictable?} Phrased another way: \textit{are there nodes in the network whose
outgoing edges can be predicted based on their most recent incoming edges? And
to what depth?}

Dependencies such as these are interesting in the context of communication
networks because they point to the presence of certain message-passing `chains'.
For example, it might be the case that whenever a node~$B$'s most recent
incoming message originated at node~$A$, the next message sent by~$B$ is likely
to be to some third node~$C$. If this is not the usual behaviour for messages
sent from~$B$, then the distribution of its outgoing edges is conditionally
dependent on its last incoming edge. We are interested in determing, for a given
network, to what extent---if any---this dependence holds. And specifically: with
respect to which nodes, which incoming edges, and to what depth.

The problem can also be stated without the need for network terminology, for
with a bit of thought one sees (Section~\ref{sec:templ soc netw}) that each
uninterrupted path within a temporal network can also be represented as a
sequence of nodes, and thus simply as a string of symbols~$x_1 x_2 \dots x_N$.
The objective then is to find substrings~$x_{n-k} \dots x_{n-1}$ that allow the
next symbol~$x_n$ to be predicted (to some extent), for we can then
regard~$x_{n-1}$ as conditionally dependent on the depth~$k-1$ substring for the
purpose of predicting the symbol~$x_n$ that follows it.

It is perhaps worth mentioning here that we are not interested in dependence on
any sort of transformed substring

The obvious probabilistic framework to adopt when approaching this problem is
that of Markov chains, however it's worth emphasising the fact that our interest
is not so much in modelling the outgoing edge distribution of each node to some
depth for the purposes of prediction, but rather in pruning the set of all
Markov states until only those that are truly useful for prediction of their
next edge remain.

on a per-case basis, whether the Markov assumption is a reasonable one; and to
what order in each case. Since independence from any directly preceding symbols
can be viewed as a Markov property of order zero, we can find significant
subsequences by assuming the Markov property throughout, and asking what order
is most likely in each case.





It would be interesting (in our humble opinion) to
know edge to terminate at~$B$ node~$B$'s incoming edges is~$A \longrightarrow
B$, and the next edge involving~$B$ is~$B \longrightarrow C$, then 

With regard to the second phrasing of the question above, it's worth emphasising
the use of the phrase `certain the fact that we are not trying to answer the question of whether a given node's
incoming edges are significant \emph{in general}, but rather \emph{which} of its
incoming neighbours (if any) are useful for prediction, in what way, and to what
depth.


\textit{are certain directed paths
within the network more predictable than others?} Or, phrased another way:
\textit{are there sequences of edges that reliably determine the edges that
follow them?} In essence, we were curious to know whether there were elements of
the data that could be said to be strongly influenced by those that directly
preceded them; and if so, to determine these elements algorithmically.

Of course the social networks that sparked the search for repetitive were simply
a particular (and convenient) way of viewing a set of sequential data---any
sequence of symbols from a given alphabet can be viewed in the same way. So the
above question should really be phrased with respect to arbitrary sequences
instead: assuming that the Markov property holds for 

\section{Methodology} %=========================================================

\subsection{Variable-order Markov models} %=====================================

\subsection{Sampling over the model space} %====================================

\section{Applications} %========================================================

\subsection{Synthetic data}\label{sec:synth dat} %==============================

\subsection{Text data}\label{sec:text dat} %====================================

\subsection{Temporal social networks}\label{sec:templ soc netw} %===============

\subsection{Genetic data}\label{sec:genet dat} %================================

\section{Implementation notes} %================================================

\end{document}